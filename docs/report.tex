\documentclass{article}

\usepackage{graphicx}
\usepackage{placeins}
\usepackage[utf8]{inputenc}
\usepackage{listings}
\usepackage{hyperref}
\usepackage{xcolor}
\usepackage[nodayofweek,level]{datetime}
\usepackage{ulem}

\definecolor{darkgreen}{RGB}{50,200,50}
\definecolor{verylightgray}{gray}{0.85}
\definecolor{goldenbrown}{rgb}{0.558215, 0.0, 0.135316}

\lstdefinelanguage{solidity}
{
	keywords={uint256, bytes32, mapping, address, uint40, bool, uint32, msg, sender, false, true},
	keywordstyle=\color{blue},
	keywords=[2]{pragma, contract, event, enum, struct, function, return, private, public, constant, returns, var, for, !, if, break, continue, throw, &&, ||, else, =, ==, <, >, <=, >=},
	keywordstyle=[2]\color{magenta},
	comment=[l]{//},
	commentstyle=\color{darkgreen},
	numbers=left,
	extendedchars=false,
	morestring=[b]",
	stringstyle=\color{goldenbrown},
	breaklines=true,
	backgroundcolor=\color{verylightgray},
	basicstyle=\linespread{1}\ttfamily\footnotesize
}

\title{Consensus in declarative process models using distributed smart-contracts}
\author{Mikkel Gaub, \\ Trondúr Høgnason, \\ Malthe Ettrup Kirkbro, \\ \& Mads Frederik Madsen }
\date{}

\begin{document}
	\begin{titlepage}
		\maketitle		
		\hspace{-18pt}
		\textit{May 15, 2017}
		\vspace{\fill}
		\section*{Abstract}
		This paper investigates how efficently declarative process models can be implemented using distributed smart-contracts, more concretely the Dynamic Condition Response (DCR) engine will be implemented on the Ethereum platform, with a focus on minimizing the cost of running such an engine.
		\thispagestyle{empty}
	\end{titlepage}
	\clearpage

	\pagenumbering{arabic}
	\setcounter{page}{1}

	\tableofcontents
	\pagebreak

	\section{Introduction}

	\section{DCR}

	Exclusion of Spawn relation.

	\section{Ethereum}
		For efficient distribution of, and consensus in, the workflows, we use the platform Ethereum. 
		Ethereum is a blockchain technology that allows for code publication and execution via the Ethereum blockchain.  
		Central for blockchain technologies is cryptocurrency, which is used to provide incentive for mining, and in Ethereum's case, to pay for computations called \textit{gas}\cite{yellow-paper}.
		Ethereum's cryptocurrency is called \textit{Ether}\cite{yellow-paper}.

		For the purposes of distribution and code execution via the Ethereum blockchain, the Ethereum yellow paper specifies a virtual machine, which achieves what the authors call "quasi-Turing-completeness"\cite{yellow-paper}. 
		They define this to be Turing-completeness, with the amount of computation bounded by the amount of gas provided,
		i.e infinite computation is impossible, since the amount of gas will always be finite. 

		\subsection{Blockchain}

		\subsection{Ethereum Virtual Machine}

		\subsection{Currency}

	\section{Implementation requirements}

	All of the proposed solutions will contain the following features:
	\begin{itemize}
		\item Creation of a workflow
		\item Execution of an activity
		\item Execution rights on the level of a user and on the level of a group
		\item Visibility?
	\end{itemize}


	\section{Multi-contract implementation}

	As the gas costs of ethereum are largely dominated by the price of creating a contract, creating a contract for each activity seems unfeasible and also has security concerns related to it which will be covered later...

	The first proposed solution is creating a contract for each workflow...

	\section{Mono-contract implementation}

	The second proposed solution is creating a single contract which controls all workflows...

	\section{Comparison}
	In order to compare the solutions to each other, a simple workflow has been created modelling each of the five relations...

	\begin{description}
		\item[Contract creation]
		\item[Successful execution] ...
		\item[Failed execution] ... 
	\end{description}

	\section{Optimizations}

		\subsection{Bitfields}

		\subsection{Incoming/Outgoing}

		\subsection{etc.}


	\section{Discussion}

	\section{Further features}

		\subsection{External relations}

		\subsection{Workflow changes}

		\subsection{External contract conditions}

	\section{Vulnerabilities}

		\subsection{External relations attack}

		\subsection{Workflow updating attack}

	\section{Conclusion}

	\pagebreak
	\addcontentsline{toc}{section}{References}	
	\begin{thebibliography}{99}

		\bibitem{test}
		Example, B.
		\textit{How to bib}.
		Penguin publishing,
		1994.

		\bibitem{yellow-paper}
		Wood, G.
		\textit{Ethereum: A Secure Decentralised Generalised Transaction Ledger}. 
		\url{http://yellowpaper.io}.
		Accessed 2017-05-09.

		\bibitem{bachelor}
		Gaub et al.
		\textit{Consensus in peer-to-peer systems}.
		Unpublished manuscript.
		IT-University of Copenhagen,
		Denmark,
		2016.


	\end{thebibliography}

	\appendix

	\section{Gas prices}

		\begin{tabular}{| l | l |}
			\hline
			Action & Cost (gas) \\ \hline
			Contract creation & 0 \\
			\hline
		\end{tabular}

	\section{Test workflow}

	\section{Multi-contract}

		\subsection{Code}

			\lstinputlisting[language=solidity]{../contracts/workflow.sol}

		\subsection{Costs}

			\begin{tabular}{| l | l |}
				\hline
				Action & Cost (gas) \\ \hline
				Contract creation & 0 \\
				\hline
			\end{tabular}

	\section{Mono-contract}

		\subsection{Code}

			\lstinputlisting[language=solidity]{../contracts/monolith.sol}

		\subsection{Costs}

			\begin{tabular}{| l | l |}
				\hline
				Action & Cost (gas) \\ \hline
				Contract creation & 0 \\
				\hline
			\end{tabular}

\end{document}