\documentclass{article}

\usepackage{graphicx}
\usepackage{placeins}
\usepackage[utf8]{inputenc}
\usepackage{listings}
\usepackage{hyperref}
\usepackage{xcolor}

\definecolor{darkgreen}{RGB}{50,200,50}
\definecolor{verylightgray}{gray}{0.85}
\definecolor{goldenbrown}{rgb}{0.558215, 0.0, 0.135316}

\lstdefinelanguage{solidity}
{
	keywords={uint256, bytes32, mapping, address, uint40, bool, uint32, msg, sender, false, true},
	keywordstyle=\color{blue},
	keywords=[2]{pragma, contract, event, enum, struct, function, return, private, public, constant, returns, var, for, !, if, break, continue, throw, &&, ||, else, =, ==, <, >, <=, >=},
	keywordstyle=[2]\color{magenta},
	comment=[l]{//},
	commentstyle=\color{darkgreen},
	numbers=left,
	extendedchars=false,
	morestring=[b]",
	stringstyle=\color{goldenbrown},
	breaklines=true,
	backgroundcolor=\color{verylightgray},
	basicstyle=\linespread{1}\ttfamily\footnotesize
}

\title{Consensus in declarative process models using distributed smart-contracts}
\author{Mikkel Gaub, \\ Tróndur Høgnason, \\ Malthe Ettrup Kirkbro, \\ \& Mads Frederik Madsen }
\date{}

\begin{document}
	\begin{titlepage}
		\maketitle		
		\hspace{-18pt}
		\textit{May 15, 2017}
		\vspace{\fill}
		\section*{Abstract}
		This paper investigates how efficently declarative process models can be implemented using distributed smart-contracts, more concretely the Dynamic Condition Response (DCR) engine will be implemented on the Ethereum platform, with a focus on minimizing the cost of running such an engine.
		\thispagestyle{empty}
	\end{titlepage}
	\clearpage

	\pagenumbering{arabic}
	\setcounter{page}{1}

	\tableofcontents
	\pagebreak

	\section{Introduction}

	\section{Distributed Condition Response graphs}
	A Distributed Condition Response (DCR) graph is a representation of a workflow. 
	The graph is made up of one or more activities with zero or more relations between them. 
	The following section is loosely based on a similar description in our bachelor-project \cite{bachelor}. 

	\subsection{Activity}
	The activities in a DCR graph have three attributes: included, executed and pending. 
	The attributes can be true or false. 
	Furthermore an activity can have role and actor specific execution rights.

	\subsubsection{Included attribute}
	If the include attribute of an activity is true, the activity is included and it can be executed. 
	If the attribute is false, the activity is excluded and can no longer be executed.

	\subsubsection{Pending attribute}
	If any activity in a workflow has a pending attribute that is true and the activity is included, the workflow is in an unfinished state. 
	Every time an activity is executed its pending attribute is set to false.
	This means that setting the pending attribute of an included activity to true is specifying that this activity \emph{must} be executed or excluded at some point to finish the workflow.

	\subsubsection{Executed attribute}
	If an activities executed attribute is false executing the activity will set its executed attribute to true.
	Executing an already executed activity will have no effect on the executed attribute.

	\begin{figure}[h]
		\centering
		\includegraphics[width=1\textwidth]{figures/activity_states.png}
	 	\caption[Activity States]
	 	{From left to right: A visual representation of an included, excluded, pending and executed activity as presented on \href{http://www.dcrgraphs.net}{dcrgraphs.net}}.
	\end{figure}

	\subsection{Relations}
	There are five types of relations which define different types of relationships between activities in a workflow. 
	These five are the \emph{condition}, \emph{response}, \emph{include}, \emph{exclude} and \emph{milestone} relations.

	\subsubsection{Condition relation}
	If there is a condition relation from activity A to activity B, then B can only be executed if A's executed attribute is true.
	\begin{figure}[h!]
		\centering
		\includegraphics[width=0.3\textwidth]{figures/ConditionRelation.png}
	 	\caption[Condition relation]
	 	{Condition relation}
	\end{figure}

	\subsubsection{Response relation}
	If there is a response relation from activity A to activity B, then B's pending attribute will be set to true every time A is executed.
	\begin{figure}[h!]
		\centering
		\includegraphics[width=0.3\textwidth]{figures/ResponseRelation.png}
	 	\caption[Response relation]
	 	{Response relation}
	\end{figure}

	\subsubsection{Include relation}
	If there is an include relation from activity A to activity B, then B's included attribute will be set to true every time A is executed.
	\begin{figure}[h!]
		\centering
		\includegraphics[width=0.3\textwidth]{figures/IncludeRelation.png}
	 	\caption[Include relation]
	 	{Include relation}
	\end{figure}

	\subsubsection{Exclude relation}
	If there is an exclude relation from activity A to activity B, then B's included attribute will be set to false every time A is executed.
	\begin{figure}[h!]
		\centering
		\includegraphics[width=0.3\textwidth]{figures/ExcludeRelation.png}
	 	\caption[Exclude relation]
	 	{Exclude relation}
	\end{figure}

	\paragraph{Milestone relation}
	If there is a milestone relation from activity A to activity B, then B can only be executed if A's pending attribute is false or A's included attribute is false.
	\begin{figure}[h!]
		\centering
		\includegraphics[width=0.3\textwidth]{figures/MilestoneRelation.png}
	 	\caption[Milestone relation]
	 	{Milestone relation}
	\end{figure}

	\subsection{Example workflow}
	Figure \ref{fig:exampleWorkflow} shows an DCR graph modeling a support ticket workflow. When the workflow is created only \texttt{Submit ticket} is executable. 
	\texttt{Close ticket} cannot be executed as it is excluded and there is a condition relation to it from an activity that is not executed. 
	\texttt{Propose solution} and \texttt{Reject solution} cannot be executed as there are condition relations to them from activities that have not been executed. 
	Lastly \texttt{Accept solution} cannot be executed as there is a milestone relation to it from \texttt{Propose solution} which is pending and not excluded.
	\begin{figure}[h!]
		\centering
		\includegraphics[width=\textwidth]{figures/exampleWorkflow.png}
	 	\caption[Example workflow]
	 	{Example workflow}
	 	\label{fig:exampleWorkflow}
	\end{figure}
	\FloatBarrier
	A typical example of an execution order of the example workflow would look like this:
	\begin{enumerate}
		\item The customer executes \texttt{Submit ticket} which excludes itself. 
		This includes \texttt{Close ticket}, but \texttt{Close ticket} is still not executable, as there is a condition relation to it from \texttt{Accept solution}.
		\texttt{Propose solution} however is now executable.
		\item The supporter executes \texttt{Propose solution}. Now \texttt{Accept solution} is executable as \texttt{Propose solution} is no longer pending and \texttt{Reject solution} is also executable as \texttt{Propose solution} is executed. 
		Furthermore \texttt{Accept solution} is now pending and must therefore be executed or excluded at some point.
		\item If the customer is not satisfied with the solution he can execute \texttt{Reject solution} which will prevent execution of \texttt{Accept solution}, as \texttt{Propose solution} is now pending and still included. 
		The workflow is now in the same state as in step 2 except from the fact that \texttt{Reject solution} is executable. 
		Executing \texttt{Reject solution} will however have no effect, as \texttt{Propose solution} is already pending.
		\item If the customer on the other hand is satisfied with the supporters solution he can execute \texttt{Accept solution}.
		Executing \texttt{Accept solution} will exclude \texttt{Propose solution} and \texttt{Reject solution}.
		This leaves \texttt{Close ticket} executable and pending.
		To leave the workflow in a finished state the supporter must execute \texttt{Close ticket}. 
	\end{enumerate}

	\section{Ethereum}

		\subsection{Blockchain}

		\subsection{Ethereum Virtual Machine}

		\subsection{Currency}

	\section{Implementation requirements}

	All of the proposed solutions will contain the following features:
	\begin{itemize}
		\item Creation of a workflow
		\item Execution of an activity
		\item Execution rights on the level of a user and on the level of a group
		\item Visibility?
	\end{itemize}


	\section{Multi-contract implementation}

	As the gas costs of ethereum are largely dominated by the price of creating a contract, creating a contract for each activity seems unfeasible and also has security concerns related to it which will be covered later...

	The first proposed solution is creating a contract for each workflow...

	\section{Mono-contract implementation}

	The second proposed solution is creating a single contract which controls all workflows...

	\section{Comparison}
	In order to compare the solutions to each other, a simple workflow has been created modelling each of the five relations...

	\begin{description}
		\item[Contract creation]
		\item[Successful execution] ...
		\item[Failed execution] ... 
	\end{description}

	\section{Optimizations}

		\subsection{Bitfields}

		\subsection{Incoming/Outgoing}

		\subsection{etc.}


	\section{Discussion}

	\section{Further features}

		\subsection{External relations}

		\subsection{Workflow changes}

		\subsection{External contract conditions}

	\section{Vulnerabilities}

		\subsection{External relations attack}

		\subsection{Workflow updating attack}

	\section{Conclusion}

	\pagebreak
	\addcontentsline{toc}{section}{References}	
	\begin{thebibliography}{99}

		\bibitem{test}
		Dr. Bib Example,
		\textit{How to bib},
		Penguin publishing,
		1994

	\end{thebibliography}

	\appendix

	\section{Gas prices}

		\begin{tabular}{| l | l |}
			\hline
			Action & Cost (gas) \\ \hline
			Contract creation & 0 \\
			\hline
		\end{tabular}

	\section{Test workflow}

	\section{Multi-contract}

		\subsection{Code}

			\lstinputlisting[language=solidity]{../contracts/workflow.sol}

		\subsection{Costs}

			\begin{tabular}{| l | l |}
				\hline
				Action & Cost (gas) \\ \hline
				Contract creation & 0 \\
				\hline
			\end{tabular}

	\section{Mono-contract}

		\subsection{Code}

			\lstinputlisting[language=solidity]{../contracts/monolith.sol}

		\subsection{Costs}

			\begin{tabular}{| l | l |}
				\hline
				Action & Cost (gas) \\ \hline
				Contract creation & 0 \\
				\hline
			\end{tabular}

\end{document}